\documentclass[12pt, a4paper]{report}
\usepackage[italian]{babel}
\usepackage[T1]{fontenc}
\usepackage[sfdefault]{noto}
\usepackage{graphicx}
\usepackage{multirow}
\usepackage{enumitem}
\usepackage{hyperref}
\hypersetup{pdfborder = 0 0 0 }
\usepackage{wrapfig}
\usepackage{color}
\linespread{1.3}
\textwidth=450pt\oddsidemargin=0pt
\begin{document}
\begin{titlepage}
\vspace{15mm}
\begin{center}
  \includegraphics{Images/uniboLogo}
\end{center}
\begin{center}
{\normalsize{\bf Corso di Laurea Magistrale in Informatica}}\\
\vspace{5mm}
{\normalsize{\bf Anno Accademico 2018/2019}}\\
\vspace{20mm}
{\Large{\bf Software Architecture}}\\
\vspace{10mm}
{\Huge{\bf Kubernetes}}\\
\vspace{25mm}
\end{center}
\begin{flushright}
{\large{Matteo Marchesini\\0000856336\\matteo.marchesini12@studio.unibo.it}}
\end{flushright}
\end{titlepage}
\tableofcontents
\chapter{Descrizione del sistema}
\begin{center}
  \includegraphics[scale = 0.9]{Images/kubernetesLogo}
\end{center}
Kubernetes è un sistema open source per la gestione di applicazioni containerizzate tra più host; fornisce un meccanismo per il deployment, la manutenzione e lo scaling di applicazioni. È stato inizialmente sviluppato dal team di Google, per poi passare nel 2015 sotto il controllo del \textit{Cloud Native Computing Foundation (CNCF)}, che attualmente lo supporta.\\ Al giorno d'oggi è uno dei sistemi di orchestrazione per applicazioni conteinerizzate più utilizzato in assoluto, con una vastità di utenti, partners e una comunità di development attiva. Non a caso tre dei quattro maggior providers di servizi Cloud - Microsoft, IBM e Google - offrono piattaforme di Container as a Service (CaaS) basate su Kubernetes. I servizi che Kubernetes mette a disposizione sono molteplici: fornisce un ambiente la gestione di container, microservizi e piattaforme cloud. Inoltre organizza l'infrastruttura di rete e di archiviazione per conto dell'utente.\\
In generale un sistema distribuito necessita di più componenti per il corretto funzionamento, alcune open source e altre commerciali; invece Kubernetes da solo fornisce uno scenario in cui le componenti lavorano insieme, andando così a formare un unico componente combinando la semplicità del Platform as a Service (PaaS) con la flessibilità dell'Infrastructure as a Service (IaaS).\\
L'orchestrazione di container ha avuto un profondo impatto in ogni aspetto del software development e deployment moderno; in particolare ha influenzato l'architettura del Platform as a Service, fornendo un aperto ed efficiente modello per il packaging, deployment, isolamento, scaling e rolling upgrade. Kubernetes svolgerà un ruolo cruciale nell'utilizzo di container da parte di imprese e start-up emergenti.
\\L'oggetto di questo report è di studiare tutto ciò che riguarda e circonda l'architettura di Kubernetes.
\chapter{Contesto}
\chapter{Drivers architetturali}
\chapter{Struttura}
\chapter{Funzioni}
\chapter{Comportamento}
\chapter{Razionale}
\chapter{Aspetti analitici}
\chapter{Stili architetturali simili o derivati}
\chapter{Referenze}
\end{document}
